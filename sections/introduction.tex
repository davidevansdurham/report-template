\section{Introduction}
\lipsum[1-2]

Important to modern day life is the expression \gls{YOLO}. 
A close second favourite expression is \gls{LOL}.
But when you use these acronyms for a second time they look like this instead: \gls{YOLO} and \gls{LOL}.
\subsection{Subsection 1}

Lorem ipsum dolor sit amet\autocite{frenkel_smit_2012}, consectetur adipiscing elit.\autocite{mutiso_electrical_2015} 
Vivamus porttitor eu ex in venenatis. 
Nullam eget risus at mauris\autocite{mutiso_electrical_2015} tristique dictum. Pellentesque malesuada, lorem vitae rutrum placerat, dolor nisi maximus ipsum, ac placerat sem ipsum nec mauris. 
Nunc eget vulputate elit, non ultricies mi. 
Vivamus a lacus quis ex suscipit hendrerit in ut mauris. 
Pellentesque fringilla in eros ut imperdiet. 
Aenean lobortis vulputate auctor. 
liquam gravida lacus ac risus eleifend tempor. 
Donec nec scelerisque magna. 
Curabitur augue enim, elementum at nibh bibendum, accumsan auctor dolor. 
Quisque in mauris dignissim, mollis purus in, congue enim. 
Quisque ut tortor rhoncus, malesuada dui vitae, varius augue.

\begin{figure}[t]
    \centering
    \includegraphics[width=\linewidth]{introduction/cat1.jpg}
    \caption[This is a caption for the list of figures]{Sed et malesuada tellus, eu ullamcorper turpis. Proin a ex vitae lectus susci}
    \label{testfig}
\end{figure}

Suspendisse potenti.
Above is Figure \ref{testfig}.
Nunc consequat nisi eget ante pulvinar, eu venenatis risus ultricies. 
Maecenas sed tristique nibh, a sodales enim. 
Sed vel erat non risus sagittis bibendum efficitur vel elit. 
Suspendisse potenti.
\begin{equation} \label{eqn1.1}
    E = mc^2 + 1 - \exp(\beta)
\end{equation}
Suspendisse euismod vel libero eget viverra. 
This is a reference to Eqn. \ref{eqn1.1}
In vitae lacus odio. Duis commodo leo vel massa rhoncus, a aliquam arcu placerat. 
In elementum velit ac malesuada tempor.

Phasellus accumsan ipsum nec lacus laoreet elementum. 
Ut convallis fermentum euismod. 
Mauris enim augue, dictum non aliquet vel, lacinia nec risus. 
Aenean volutpat est quis massa faucibus, a faucibus ante facilisis. 
\begin{equation} \label{eqn1.2}
    \sigma = \sigma_s + \sigma_d
\end{equation}
Donec vehicula quam magna, vitae euismod diam congue in. 
This is another reference, but this time to Eqn. \ref{eqn1.2}.
Quisque in ultricies velit, ut dignissim neque. 
Donec sollicitudin mauris euismod odio dapibus, at accumsan risus fermentum. 
Morbi tempor vel neque sodales fringilla.

Cras imperdiet massa ipsum. 
Curabitur feugiat nibh ut sollicitudin feugiat. 
Curabitur consequat libero ac magna dapibus imperdiet. 
Sed convallis diam sit amet lorem dignissim, et commodo erat tempor. 
Praesent est libero, pharetra in euismod non, sodales vel lectus.
Somewhere near here is Figure \ref{testfig2}. 
Cras vitae eros nec justo sodales lobortis.
\begin{figure}[t]
    \centering
    \includegraphics[width=\linewidth]{introduction/cat1.jpg}
    \caption[This is another cat]{Sed et malesuada tellus, eu ullamcorper turpis. Proin a ex vitae lectus susci}
    \label{testfig2}
\end{figure} 
Quisque hendrerit neque scelerisque leo vestibulum feugiat eget vel ante. 
Donec ac porta tortor, in mollis ante. 
Duis vulputate, velit nec porttitor placerat, eros tortor faucibus eros, ut tristique lacus enim sit amet orci. 
Aliquam rutrum ipsum non sapien tempor, vel suscipit libero elementum. 
Class aptent taciti sociosqu ad litora torquent per conubia nostra, per inceptos himenaeos. 
Ut magna nisi, elementum non turpis in, dictum euismod nibh.

Lorem ipsum dolor sit amet, consectetur adipiscing elit. 
Sed non eleifend ipsum, nec elementum urna. 
Donec eget varius quam \todo{This is an example of a todo note}. Nunc tempus sit amet ex vel congue. 
Morbi lectus ante, volutpat eget felis non, accumsan suscipit massa. 
Phasellus dictum orci a dui placerat, non ultrices neque ornare. 
Cras posuere interdum tellus, quis elementum erat tincidunt eu. 
Suspendisse maximus congue maximus.

\subsection{Subsection 2}
\lipsum[5-8]